\documentclass[addpoints,answers]{beamer}

\usepackage{graphicx}
\usepackage{xcolor}
\usepackage[utf8]{inputenc}
\usepackage[croatian]{babel}
\usepackage[T1]{fontenc}
\usepackage{mathexam}
\usepackage{hyperref}
%Jebeno kad dodam package i ove commande dole mi vise ne radi nista kako piše u literaturi, al mi ne daje errore tako da nez sta se desava
\newcounter{points}
\newenvironment{choices}{\begin{enumerate}}{\end{enumerate}}
\newenvironment{questions}{\setcounter{points}{0}}{}
\newcommand{\question}[1][1]{\addtocounter{points}{#1}}


\begin{document}
	\title{Izrada ispita}
	\date{27.1.2019.}
	\author{Renato Štrbac, Dominik Premelč, Luka Županović}
	
	
	\pagenumbering{arabic}
	\maketitle
	\tableofcontents
	
	%
	%
	%Štrbac - dio
	%
	%
\begin{frame}
	\begin{center}
		\textbf{Osnovne komande za izradu ispita u LaTeX-u}
	\end{center}
\end{frame}

\begin{frame}
Kako bi započeli s pisanjem prvo moramo postaviti oblik koji se koristi za pisanje ispita u LaTeX-u:
\newline
\newline
\color{blue}
\textbackslash{documentclass[addpoints,answers]\{exam\}}
\color{black}
\newline
\newline
Kako bi stvorili prostor za pisanje(Npr. za pisanje imena i prezimena na početku testa) koristimo naredbu \textbackslash{makebox}
\newline
\newline
\color{blue}
\textbackslash{makebox[\textbackslash{textwidth}]}\{Ime i prezime:\textbackslash{enspace\textbackslash{hrulefill}}\}
\color{black}
\newline
\newline
\textbackslash{textwidth} postavlja veličinu prostora za pisanje na veličinu teksta dok \textbackslash{enspace\textbackslash{hrulefill}} ispunjuje ostatak retka vodoravnom crtom
\end{frame}

\begin{frame}
Kada tu naredbu upišemo trebali bi smo dobiti ovo:
\newline
\newline
\makebox[\textwidth]{Ime i prezime:\enspace\hrulefill}
\end{frame}

\begin{frame}
Kako bi postavili pitanja za ispit u LaTeX-u koristimo okruženje:
\newline
\newline
\color{blue}
\textbackslash{begin\{questions\}}
\color{black}
\color{blue}
\newline
\newline
\textbackslash{end\{questions\}}
\color{black}
\newline
\newline
Zatim dodajemo tekst ispod naredbi:
\newline
\newline
\color{blue}
\textbackslash{question[1]}
\color{black}
\newline
Koliko je 2+2?
\newline
\newline
\color{blue}
\textbackslash{question[4]}
\color{black}
\newline
Tko je pobijedio drugi svjetski rat?

\end{frame}


\begin{frame}
\begin{questions}
	
To izgleda ovako:
\newline
\newline
\question[1]
[1 Point] Koliko je 2+2?
\newline
\newline
\question[4]
[4 Points] Tko je pobijedio drugi svjetski rat?
\end{questions}
\end{frame}




\begin{frame}
Za postavljanje podpitanja koristi se okruženje:
\newline
\newline
\color{blue}
\textbackslash{begin\{parts\}}
\color{black}
\newline
\newline
\color{blue}
\textbackslash{end\{parts\}}
\color{black}
\newline
\newline
Unutar tog okruženja koristimo
\newline
\newline
\color{blue}
\textbackslash{part}
\color{black}
\newline
\newline
za svako podpitanje koje želimo postavit
\end{frame}

\begin{frame}
\begin{questions} 
	\question[1]
 [1 Point] Koliko je 2+2?\newline
	(a) 3 \newline
	(b)	4 
\end{questions}
\end{frame}


\begin{frame}
	Koristimo
	\newline
	\newline
	\color{blue}
	\textbackslash{vspace\{2in\}} 
	\color{black}
	\newline
	\newline
	 kako bi ostavili mjesto za pisanje. Ovdje je postavljeno mjesto od 2 inča, ali ne mora nužno biti.
\end{frame}

\begin{frame}
	Po standardu su bodovi pokraj pitanja uključeni, ali ako mi želimo ih možemo isključiti sa naredbom
	\newline
	\newline
	\color{blue}
	\textbackslash{noaddpoints}
	\color{black}
	\newline
	\newline
	A ako ih želimo vratiti koristimo 
	\newline
	\newline
	\color{blue}
	\textbackslash{addpoints}
	\color{black}
\end{frame}

\begin{frame}
	Isto tako koristimo slijedeće naredbe da odredimo di će se bodovi nalazit. Ako ništa ne odredimo po standardu bodovi će pisat ispred pitanja.
	\newline
	\newline
	\color{blue}
	\textbackslash{pointsinmargin}
		\color{black}
	(U lijevoj margini)
		\color{blue}
	\newline
	\newline
	\textbackslash{pointsinrightmargin}
		\color{black}
	(U desnoj margini)
		\color{blue}
	\newline
	\newline
	\textbackslash{pointstwosided}
		\color{black}
	(U desnoj margini na neparnim, a u lijevoj na parnim stranicama)
		\color{blue}
	\newline
	\newline
	\textbackslash{pointstwosidedreversed}
		\color{black}
	(U lijevoj margini na neparnim, a u desnoj na parnim stranicama)
		\color{blue}
	\newline
	\newline
	\textbackslash{nopointsinmargin}\color{black}   (Da se vratimo na standard)
\end{frame}

\begin{frame}
	Ako koristimo:
	\newline
	\newline
	\color{blue}
	\textbackslash{pointsinmargin}
	\newline
	\newline
	\textbackslash{pointsinrightmargin}
	\newline
	\newline
	\color{black}
	Onda možemo mijenjat veličinu margina, npr.
	\newline
	\newline
	\color{blue}
	\textbackslash{setlength\{\textbackslash{marginpointssep}\}}\{5pt\}
	\newline
	\newline
	\textbackslash{setlength\{\textbackslash{rightpointsmargin}\}}\{1cm\}
	\color{black}
	\newline
	\newline
	Duljine su naravno proizvoljne.
\end{frame}


\begin{frame}
	Postoje i razne naredbe kojima možemo ukrasiti bodove. Npr.
	\newline
	\newline
	\color{blue}
	\textbackslash{boxedpoints}
	\color{black}
	(Stavi bodove u pravokutnik)
	\color{blue}
	\newline
	\newline
	\color{blue}
	\textbackslash{bracketed points}
	\color{black}
	(Stavi ih u zagrade)
	\newline
	\newline
	Osim toga po standardu se bodovi označuju engleskom riječi "Point". Ako želimo promijeniti i napisat recimo "Bod" pisali bi:
	\newline
	\newline
	\color{blue}
	\textbackslash{pointpoints\{Bod\}\{Bodovi\}}
	\color{black}
	\newline
	(U lijevoj zagradi pišemo jedninu, a u desnoj množinu)
	\color{blue}
\end{frame}

\begin{frame}
	Ako želimo postaviti naslov ispod kojeg čemo redati pitanja možemo upotrijebiti naredbu:
	\newline
	\newline
	\color{blue}
	\textbackslash{titledquestion\{Design\}[10]}
	\color{black}
	\newline
	\newline
	Ovo će postaviti naslov na "Design" i postaviti broj 10 pokraj naslova koji opisuje broj bodova.
\end{frame}



%Note, izostavi ista vezano uz bonus questions ako ima u tvom djelu, ja sam to izostavio u svom
%
%Premelč - dio
%
%

\begin{frame}

\end{frame}

%
%	
%Županović - dio
%
%

\begin{frame}

\end{frame}


\end{document}
	