\documentclass{beamer}
\usepackage{graphicx}
\usepackage{xcolor}
\usepackage[utf8]{inputenc}
\usepackage[croatian]{babel}
\usepackage[T1]{fontenc}

\title{Izrada ispita}
\date{27.1.2019.}
\author{Renato Štrbac, Dominik Premelč, Luka Županović}

\begin{document}
\pagenumbering{arabic}
\maketitle
\tableofcontents

%
%
%Štrbac - dio
%
%
\begin{frame}
	\begin{center}
		\textbf{Osnovne komande za izradu ispita u LaTeX-u}
	\end{center}
\end{frame}

\begin{frame}
Kako bi započeli s pisanjem prvo moramo postaviti oblik koji se koristi za pisanje ispita u LaTeX-u:
\newline
\newline
\color{blue}
\textbackslash{documentclass\{exam\}}
\color{black}
\newline
\newline
Kako bi stvorili prostor za pisanje(Npr. za pisanje imena i prezimena na početku testa) koristimo naredbu \textbackslash{makebox}
\newline
\newline
\color{blue}
\textbackslash{makebox[\textbackslash{textwidth}]}\{Ime i prezime:\textbackslash{enspace\textbackslash{hrulefill}}\}
\color{black}
\newline
\newline
\textbackslash{textwidth} postavlja veličinu prostora za pisanje na veličinu teksta dok \textbackslash{enspace\textbackslash{hrulefill}} ispunjuje ostatak retka vodoravnom crtom
\end{frame}

\begin{frame}
Kada tu naredbu upišemo trebali bi smo dobiti ovo:
\newline
\newline
\makebox[\textwidth]{Ime i prezime:\enspace\hrulefill}
\end{frame}

\begin{frame}
Kako bi postavili pitanja za ispit u LaTeX-u koristimo okruženje:
\newline
\newline
\color{blue}
\textbackslash{begin\{questions\}}
\color{black}
\color{blue}
\newline
\newline
\textbackslash{end\{questions\}}
\color{black}
\newline
\newline
Zatim dodajemo tekst ispod naredbi:
\newline
\newline
\color{blue}
\textbackslash{question[1 bod]}
\color{black}
\newline
Koliko je 2+2?
\newline
\newline
\color{blue}
\textbackslash{question[3 boda]}
\color{black}
\newline
Tko je pobijedio drugi svjetski rat?

\end{frame}

\begin{frame}
	To izgleda ovako:
	\newline
	\newline
	\question[1 bod]
	Koliko je 2+2?
	\newline
	\newline
	\question[3 boda]
	Tko je pobijedio drugi svjetski rat?
\end{frame}


\begin{frame}
Za postavljanje podpitanja koristi se okruženje:
\newline
\newline
\color{blue}
\textbackslash{begin\{parts\}}
\color{black}
\newline
\newline
\color{blue}
\textbackslash{end\{parts\}}
\color{black}
\newline
\newline
Unutar tog okruženja koristimo
\newline
\newline
\color{blue}
\textbackslash{part}
\color{black}
\newline
\newline
za svako podpitanje koje želimo postavit
\end{frame}

\begin{frame}
1. 
	\question[1 bod]
	Koliko je 2+2?
	\part
		\newline
		(a) 3 \newline
		(b) 4

\end{frame}




%
%
%Premelč - dio
%
%

\begin{frame}

\end{frame}

%
%	
%Županović - dio
%
%

\begin{frame}

\end{frame}
	

	
\end{document}