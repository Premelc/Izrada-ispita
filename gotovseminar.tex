\documentclass[addpoints,answers]{beamer}

\usepackage{graphicx}
\usepackage{biblatex}
\usepackage{xcolor}
\usepackage[utf8]{inputenc}
\usepackage[croatian]{babel}
\usepackage[T1]{fontenc}
\usepackage{mathexam}
\usepackage{hyperref}
% kad dodam package i ove commande dole mi vise ne radi nista kako piše u literaturi, al mi ne daje errore tako da nez sta se desava
\newcounter{points}
\newenvironment{choices}{\begin{enumerate}}{\end{enumerate}}
\newenvironment{questions}{\setcounter{points}{0}}{}
\newcommand{\question}[1][1]{\addtocounter{points}{#1}}


\begin{document}
	\title{Izrada ispita}
	\date{27.1.2019.}
	\author{Renato Štrbac, Dominik Premelč, Luka Županović}
	
	
	\pagenumbering{arabic}
	\maketitle
	\tableofcontents
	
	%
	%
	%Štrbac - dio
	%
	%
	\begin{frame}
	\begin{center}
		\textbf{Osnovne komande za izradu ispita u LaTeX-u}
	\end{center}
\end{frame}

\begin{frame}
Kako bi započeli s pisanjem prvo moramo postaviti oblik koji se koristi za pisanje ispita u LaTeX-u:
\newline
\newline
\color{blue}
\textbackslash{documentclass[addpoints,answers]\{exam\}}
\color{black}
\newline
\newline
Kako bi stvorili prostor za pisanje(Npr. za pisanje imena i prezimena na početku testa) koristimo naredbu \textbackslash{makebox}
\newline
\newline
\color{blue}
\textbackslash{makebox[\textbackslash{textwidth}]}\{Ime i prezime:\textbackslash{enspace\textbackslash{hrulefill}}\}
\color{black}
\newline
\newline
\textbackslash{textwidth} postavlja veličinu prostora za pisanje na veličinu teksta dok \textbackslash{enspace\textbackslash{hrulefill}} ispunjuje ostatak retka vodoravnom crtom
\end{frame}

\begin{frame}
Kada tu naredbu upišemo dobijemo ovo:
\newline
\newline
\makebox[\textwidth]{Ime i prezime:\enspace\hrulefill}
\end{frame}

\begin{frame}
Kako bi postavili pitanja za ispit u LaTeX-u koristimo okruženje:
\newline
\newline
\color{blue}
\textbackslash{begin\{questions\}}
\color{black}
\color{blue}
\newline
\newline
\textbackslash{end\{questions\}}
\color{black}
\newline
\newline
Zatim dodajemo tekst ispod naredbi:
\newline
\newline
\color{blue}
\textbackslash{question[1]}
\color{black}
\newline
Koliko je 2+2?
\newline
\newline
\color{blue}
\textbackslash{question[4]}
\color{black}
\newline
Tko je pobijedio drugi svjetski rat?

\end{frame}


\begin{frame}
\begin{questions}

To izgleda ovako:
\newline
\newline
\question[1]
[1 Point] Koliko je 2+2?
\newline
\newline
\question[4]
[4 Points] Tko je pobijedio drugi svjetski rat?
\end{questions}
\end{frame}




\begin{frame}
Za postavljanje podpitanja koristi se okruženje:
\newline
\newline
\color{blue}
\textbackslash{begin\{parts\}}
\color{black}
\newline
\newline
\color{blue}
\textbackslash{end\{parts\}}
\color{black}
\newline
\newline
Unutar tog okruženja koristimo
\newline
\newline
\color{blue}
\textbackslash{part}
\color{black}
\newline
\newline
za svako podpitanje koje želimo postavit
\end{frame}

\begin{frame}
\begin{questions} 
\question[1]
[1 Point] Koliko je 2+2?\newline
(a) 3 \newline
(b)	4 
\end{questions}
\end{frame}


\begin{frame}
Koristimo
\newline
\newline
\color{blue}
\textbackslash{vspace\{2in\}} 
\color{black}
\newline
\newline
kako bi ostavili mjesto za pisanje. Ovdje je postavljeno mjesto od 2 inča, ali ne mora nužno biti.
\end{frame}

\begin{frame}
Po standardu su bodovi pokraj pitanja uključeni, ali se mogu isključiti sa naredbom
\newline
\newline
\color{blue}
\textbackslash{noaddpoints}
\color{black}
\newline
\newline
A ako ih želimo vratiti koristimo 
\newline
\newline
\color{blue}
\textbackslash{addpoints}
\color{black}
\end{frame}

\begin{frame}
Isto tako koristimo slijedeće naredbe da odredimo di će se bodovi nalazit. Ako ništa ne odredimo po standardu bodovi će pisat ispred pitanja.
\newline
\newline
\color{blue}
\textbackslash{pointsinmargin}
\color{black}
(U lijevoj margini)
\color{blue}
\newline
\newline
\textbackslash{pointsinrightmargin}
\color{black}
(U desnoj margini)
\color{blue}
\newline
\newline
\textbackslash{pointstwosided}
\color{black}
(U desnoj margini na neparnim, a u lijevoj na parnim stranicama)
\color{blue}
\newline
\newline
\textbackslash{pointstwosidedreversed}
\color{black}
(U lijevoj margini na neparnim, a u desnoj na parnim stranicama)
\color{blue}
\newline
\newline
\textbackslash{nopointsinmargin}\color{black}   (Da se vratimo na standard)
\end{frame}

\begin{frame}
Ako koristimo:
\newline
\newline
\color{blue}
\textbackslash{pointsinmargin}
\newline
\newline
\textbackslash{pointsinrightmargin}
\newline
\newline
\color{black}
Onda možemo mijenjat veličinu margina, npr.
\newline
\newline
\color{blue}
\textbackslash{setlength\{\textbackslash{marginpointssep}\}}\{5pt\}
\newline
\newline
\textbackslash{setlength\{\textbackslash{rightpointsmargin}\}}\{1cm\}
\color{black}
\newline
\newline
Duljine su naravno proizvoljne.
\end{frame}


\begin{frame}
Postoje i razne naredbe kojima možemo ukrasiti bodove. Npr.
\newline
\newline
\color{blue}
\textbackslash{boxedpoints}
\color{black}
(Stavi bodove u pravokutnik)
\color{blue}
\newline
\newline
\color{blue}
\textbackslash{bracketed points}
\color{black}
(Stavi ih u zagrade)
\newline
\newline
Osim toga po standardu se bodovi označuju engleskom riječi "Point". Ako želimo promijeniti i napisat recimo "Bod" pisali bi:
\newline
\newline
\color{blue}
\textbackslash{pointpoints\{Bod\}\{Bodovi\}}
\color{black}
\newline
(U lijevoj zagradi pišemo jedninu, a u desnoj množinu)
\color{blue}
\end{frame}

\begin{frame}
Ako želimo postaviti naslov ispod kojeg čemo redati pitanja možemo upotrijebiti naredbu:
\newline
\newline
\color{blue}
\textbackslash{titledquestion\{Design\}[10]}
\color{black}
\newline
\newline
Ovo će postaviti naslov na "Design" i postaviti broj 10 pokraj naslova koji opisuje broj bodova.
\end{frame}



%Note, izostavi ista vezano uz bonus questions ako ima u tvom djelu, ja sam to izostavio u svom
%
%Premelč - dio
%
%

\begin{frame}
U LaTex-u imamo mogućnost upravljanja marginama pomoću naredbi:
\newline
\newline
\color{blue}
\textbackslash{uplevel}
\newline
\newline
\color{black}
\color{blue}
\textbackslash{fullwidth}
\color{black}
\newline
\newline
Naredba \color{blue}\textbackslash{uplevel} \color{black} smanjuje marginu s lijeve strane za jednu razinu dok će naredba \color{blue} \textbackslash{fullwidth} \color{black} smanjiti margine na naćin da tekst ispisuje od ruba do ruba stranice.
\end{frame}


\begin{frame}

To izgleda ovako:
%NEDOVRŠENO
%mjesto za priumjer \uplevel komande
\end{frame}

\begin{frame}
%NEDOVRŠENO
%mjesto za pruimjer \fullwidth komande
\end{frame}

\begin{frame}
Često koristimo komandu \color{blue} \textbackslash{fullwidth} \color{black} kako bi naslovili djelove ispita
\newline
\newline
Isto se može postići koristeći komande \color{blue} \textbackslash{part}\color{black} \newline 
i komandu \color{blue}\textbackslash{section}\color{black}
\end{frame}

\begin{frame}
%Ovdje nes ne stima,kao da nemamo ukljucen exam paket,nek neko proba ovo pokrenut u TexStudio jer ja radim na Overleafu
Koristeći komandu \color{blue}\textbackslash{fullwidth}\color{black}:
\newline\newline
\begin{questions}
\question
1. Objasnite fotoelektrični učinak.\newline\newline
\fullwidth{\Large \textbf{Esejska pitanja}}\newline\newline
\question
2. U obliku eseja objasnite što vas je motiviralo da upišete računarstvo.\newline\newline
\fullwidth{\Large \textbf{Laboratorijska vježba}}\newline\newline
\question
3. Izmjerite struju i napon na kondenzatoru C1.
\end{questions}

\end{frame}

\begin{frame}{Ostavljanje mjesta za odgovore}
Postoji nekoliko naćina za ostaviti mjesto za odgovor:
\begin{enumerate}
\item Prazno mjesto (\color{blue}\textbackslash{solution}\color{black})
\item Prazni okvir (\color{blue}\textbackslash{solutionorbox}\color{black})
\item Prostor ispunjen linijama (\color{blue}\textbackslash{solutionorlines}\color{black})
\item Prostor ispunjen linijama točaka (\color{blue}\textbackslash{solutionordottedlines}\color{black})
\item Prostor ispunjen mrežom (\color{blue}\textbackslash{solutionorgrid}\color{black})
\newline\newline
\end{enumerate}
Ove komande podrazumjevaju otvaranje različitih okruženja za pojedini odgovor \newline
U nastavku čemo proučiti kako ostaviti mjersto za odgovor bbez otvaranja novog okruženja
%dodati primjer za svako?
\end{frame}

\begin{frame}{Ostavljanje praznog mjesta za odgovor}

Koristimo naredbu \color{blue}\textbackslash{vspace*\{1in\}}\color{black}
\newline
Broj u zagradi predstavlja veličinu praznog mjesta koje želimo ostaviti za odgovor
\newline
Također možemo koristiti naredbu \color{blue}\textbackslash{vspace*\{stretch\{1\}\}}\color{black} nakon svakog pitanja kako bi ravnomjerno rasporedili prazan prostor na stranici

\end{frame}

\begin{frame}{Ostavljanje praznog okvira za odgovor}
Koristimo naredbu \color{blue}\textbackslash{makeemptybox\{length}\color{black}
\newline
Ovako stvorenom praznom okviru kasnije možemo mjenjati boju tako da u preambuli dokumenta navedemo paket \color{blue}\textbackslash{usepackage\{color\}}\color{black}
\newline
\newline
Zatim moramo defrinrati boju komandom\newline
\color{blue}\textbackslash{definecolor\{SolutionBoxColor\}\{gray\}\{0.8\}}\color{black}
\newline
\newline
Nakon što smo definirali boju,okvire obojimo komandom \color{blue}\textbackslash{colorsolutionboxes}\color{black}

\end{frame}

\begin{frame}{Ostavljanje linija za duzi odgovor}

Korsiti se komanda \color{blue}\textbackslash{fillwithlines\{length\}}\color{black}\newline\newline
Te linije se mogu obojati na isti naćin kao i prazni okviri to jest, komandom \color{blue}\textbackslash{colorfillwithlines}\color{black}
\newline
Udaljenost između linija određujemo komandom\color{blue}\textbackslash{setlength}\textbackslash{linefillheight\{.25in\}}\color{black}\newline
Zakođer možemo mjenjati i debljinu linija pomoću komande\color{blue} \textbackslash{setlength}\textbackslash{linefillthickness\{0.1pt\}}
\end{frame}

\begin{frame}{Ostavljanje linija točaka za odgovor}
Koristimo naredbu  \color{blue}\textbackslash{fillwithdottedlines\{length\}}\color{black}\newline
Ovim linijama također možemo mjenjati boju i udaljenost između točaka na isti način kao i kod prijašnjih primjera

\end{frame}

\begin{frame}{Ostavljanje mjesta za odgovor popunjeno mrežom}
Koristimo naredbu \color{blue}\textbackslash{fillwithgrid\{lenght\}}\color{black}\newline
veličinu mreže i razmaka između linija mreže definiramo komandama \color{blue}\textbackslash{setlenght\{\textbackslash{gridsize}\}\{5mm\}}\color{black} te \color{blue}\textbackslash{setlenght\{\textbackslash{gridlinewidth}\}\{0.1pt\}}\color{black}
\newline
Također mrežu možemo obojati na isti način kao i prethodne primjere

\end{frame}
\begin{frame}{Korištenje mreže}
Mrežu također možemo staviti preko cjele stranice kako bi stvorili papir na kockice
\newline
To postižemo uz pomoć komande\color{blue}\textbackslash{fillwithgrid\{\textbackslash{stretch\{1\}}}
%dodati primjer?
\end{frame}




%
%	
%Županović - dio
%
%

\begin{frame}{Tablice ocjena}
Mogu biti indeksirane po broju pitanja ili broju stranice
\newline
Komanda za kreiranje tablice \color{blue}
\textbackslash{gradetable}
\newline
\color{black}
A ako je indeksirana po broju pitanja ili stranice \color{blue} \textbackslash{pointtable}
\color{black}
\newline
Mogući dodatni argumenti su [v] or [h] za okomitu ili vodoravnu tablicu
\end{frame}

\begin{frame}{Mijenjanje margine stranice}
Moguće klase dokumenata \color{blue} a4paper, a5paper, b5paper, letterpaper, legalpaper, executivepaper, landscape\color{black}
\newline
Za povećanje margina koristi se naredba \color{blue} \textbackslash{extrawidth}
\color{black}
\newline
Primjer \color{blue}\textbackslash{extrawidth\{-1in\}} \color{black}
pomiče lijevu i desnu marignu za pola incha


\end{frame}

\begin{frame}
Ukoliko se tekst pitanja nalazi pri dnu stranice kombinacija naredbi \color{blue} \textbackslash{ifcontinuation} \color{black} i \color{blue} \textbackslash{ContinuedQuestion} \color{black}nam omogućuje da dio teksta ode u zaglavlje iduće stranice

\end{frame}
\begin{frame}
\begin{center}
     \color{red} {\huge KRAJ}
\end{center}


\end{frame}

\begin{thebibliography}{2}
\bibitem{wiki} 

\\\texttt{ https://en.wikibooks.org/wiki/LaTeX/Teacher}

 
\bibitem{MIT Stranica}


\\\texttt{http://www-math.mit.edu/~psh/exam/examdoc.pdf}
\end{thebibliography}


\end{document}

